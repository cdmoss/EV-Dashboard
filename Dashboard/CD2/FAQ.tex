\documentclass[]{article}
\usepackage{lmodern}
\usepackage{amssymb,amsmath}
\usepackage{ifxetex,ifluatex}
\usepackage{fixltx2e} % provides \textsubscript
\ifnum 0\ifxetex 1\fi\ifluatex 1\fi=0 % if pdftex
  \usepackage[T1]{fontenc}
  \usepackage[utf8]{inputenc}
\else % if luatex or xelatex
  \ifxetex
    \usepackage{mathspec}
  \else
    \usepackage{fontspec}
  \fi
  \defaultfontfeatures{Ligatures=TeX,Scale=MatchLowercase}
\fi
% use upquote if available, for straight quotes in verbatim environments
\IfFileExists{upquote.sty}{\usepackage{upquote}}{}
% use microtype if available
\IfFileExists{microtype.sty}{%
\usepackage{microtype}
\UseMicrotypeSet[protrusion]{basicmath} % disable protrusion for tt fonts
}{}
\usepackage[margin=1in]{geometry}
\usepackage{hyperref}
\hypersetup{unicode=true,
            pdfborder={0 0 0},
            breaklinks=true}
\urlstyle{same}  % don't use monospace font for urls
\usepackage{graphicx,grffile}
\makeatletter
\def\maxwidth{\ifdim\Gin@nat@width>\linewidth\linewidth\else\Gin@nat@width\fi}
\def\maxheight{\ifdim\Gin@nat@height>\textheight\textheight\else\Gin@nat@height\fi}
\makeatother
% Scale images if necessary, so that they will not overflow the page
% margins by default, and it is still possible to overwrite the defaults
% using explicit options in \includegraphics[width, height, ...]{}
\setkeys{Gin}{width=\maxwidth,height=\maxheight,keepaspectratio}
\IfFileExists{parskip.sty}{%
\usepackage{parskip}
}{% else
\setlength{\parindent}{0pt}
\setlength{\parskip}{6pt plus 2pt minus 1pt}
}
\setlength{\emergencystretch}{3em}  % prevent overfull lines
\providecommand{\tightlist}{%
  \setlength{\itemsep}{0pt}\setlength{\parskip}{0pt}}
\setcounter{secnumdepth}{0}
% Redefines (sub)paragraphs to behave more like sections
\ifx\paragraph\undefined\else
\let\oldparagraph\paragraph
\renewcommand{\paragraph}[1]{\oldparagraph{#1}\mbox{}}
\fi
\ifx\subparagraph\undefined\else
\let\oldsubparagraph\subparagraph
\renewcommand{\subparagraph}[1]{\oldsubparagraph{#1}\mbox{}}
\fi

%%% Use protect on footnotes to avoid problems with footnotes in titles
\let\rmarkdownfootnote\footnote%
\def\footnote{\protect\rmarkdownfootnote}

%%% Change title format to be more compact
\usepackage{titling}

% Create subtitle command for use in maketitle
\providecommand{\subtitle}[1]{
  \posttitle{
    \begin{center}\large#1\end{center}
    }
}

\setlength{\droptitle}{-2em}

  \title{}
    \pretitle{\vspace{\droptitle}}
  \posttitle{}
    \author{}
    \preauthor{}\postauthor{}
    \date{}
    \predate{}\postdate{}
  
\usepackage{booktabs}
\usepackage{longtable}
\usepackage{array}
\usepackage{multirow}
\usepackage{wrapfig}
\usepackage{float}
\usepackage{colortbl}
\usepackage{pdflscape}
\usepackage{tabu}
\usepackage{threeparttable}
\usepackage{threeparttablex}
\usepackage[normalem]{ulem}
\usepackage{makecell}
\usepackage{xcolor}

\begin{document}

\hypertarget{questions}{%
\subsubsection{Questions}\label{questions}}

\hypertarget{consumer}{%
\paragraph{Consumer}\label{consumer}}

\hypertarget{how-much-does-an-e-golf-cost}{%
\subparagraph{How much does an e-golf
cost?}\label{how-much-does-an-e-golf-cost}}

A 2019 Volkswagen E-golf starts at \$36,720.

\hypertarget{how-much-does-it-cost-to-charge-the-e-golf}{%
\subparagraph{How much does it cost to charge the
e-golf?}\label{how-much-does-it-cost-to-charge-the-e-golf}}

\begin{tabular}{l|r}
\hline
cost & power\_usage\\
\hline
\$2.63 & 41.422\\
\hline
\end{tabular}

\hypertarget{what-maintenance-do-electric-vehicles-require}{%
\subparagraph{What maintenance do electric vehicles
require?}\label{what-maintenance-do-electric-vehicles-require}}

\begin{itemize}
\tightlist
\item
  Battery
\item
  Brakes
\item
  Tires
\end{itemize}

\hypertarget{what-features-does-the-e-golf-have-i.e.-departure-maybe-a-broad-question-we-could-split-it-up-and-instead-ask-how-each-feature-works}{%
\subparagraph{What features does the e-golf have (i.e.~departure, maybe
a broad question, we could split it up and instead ask how each feature
works?)?}\label{what-features-does-the-e-golf-have-i.e.-departure-maybe-a-broad-question-we-could-split-it-up-and-instead-ask-how-each-feature-works}}

\hypertarget{what-type-of-warranty-does-the-e-golf-have}{%
\subparagraph{What type of warranty does the e-golf
have?}\label{what-type-of-warranty-does-the-e-golf-have}}

Basic: 3 yrs/60,000 km Battery: 8 yrs/160,000 km Drive-Train: 5
yrs/100,000 km Corrosion: 12 yrs/unlimited km Roadside Assistance: 3
yrs/unlimited km Maintenance 2 yrs/40,000 km

\hypertarget{what-type-of-incentives-come-with-owning-an-electric-vehicle}{%
\subparagraph{What type of incentives come with owning an electric
vehicle?}\label{what-type-of-incentives-come-with-owning-an-electric-vehicle}}

As per the iZEV program, the Canadian Federal Government will award
rebates for the following amounts (pertaining to the 2019 VW E-golf):

\begin{itemize}
\tightlist
\item
  \$5,000 for full purchase
\item
  \$1,250 for a 12-month lease
\item
  \$2,500 for a 24-month lease
\item
  \$3,750 for a 36-month lease
\end{itemize}

(\href{https://www.tc.gc.ca/en/services/road/innovative-technologies/list-eligible-vehicles-under-izev-program.html}{list
of vehicles eligible for the iZEV program}).

\hypertarget{charging}{%
\paragraph{Charging}\label{charging}}

\hypertarget{how-long-does-it-take-to-charge-on-average}{%
\subparagraph{How long does it take to charge on
average?}\label{how-long-does-it-take-to-charge-on-average}}

There are typically 3 levels of charger:

\begin{itemize}
\item
  Level 1 (120V): full charge takes 8 - 12hrs
\item
  Level 2 (240V): full charge takes 4 - 6hrs
\item
  Level 3 (480V): full charge takes 30min
\end{itemize}

\hypertarget{what-are-the-options-for-home-charging-stations-and-how-much-do-they-cost}{%
\subparagraph{What are the options for home charging stations and how
much do they
cost?}\label{what-are-the-options-for-home-charging-stations-and-how-much-do-they-cost}}

The E-golf comes with a 120V level 1 charger that is compatible with a
standard

\hypertarget{which-chargers-are-compatible-with-the-e-golf-how-many-are-there}{%
\subparagraph{Which chargers are compatible with the e-golf? How many
are
there?}\label{which-chargers-are-compatible-with-the-e-golf-how-many-are-there}}

J1772 - regular speed CCS - fast charging

\hypertarget{how-much-electricity-does-the-e-golf-use-cost-per-kms}{%
\subparagraph{How much electricity does the e-golf use? Cost per
KMs}\label{how-much-electricity-does-the-e-golf-use-cost-per-kms}}

Average cost per Km is \$0.01.

\hypertarget{driving}{%
\paragraph{Driving}\label{driving}}

\hypertarget{what-is-the-averge-range-of-the-vehicle}{%
\subparagraph{What is the averge range of the
vehicle?}\label{what-is-the-averge-range-of-the-vehicle}}

\begin{verbatim}
## # A tibble: 5 x 2
##   Temperature    `Max Range`
##   <chr>                <dbl>
## 1 -35.1 to -20.8         107
## 2 -20.8 to -6.6          175
## 3 -6.6 to 7.6            240
## 4 7.6 to 21.8            289
## 5 21.8 to 36.1           296
\end{verbatim}

\hypertarget{what-is-regenerative-braking-how-do-the-different-brake-modes-d-d1-d2-d3-b-effect-the-driving-experience-of-the-vehicle}{%
\subparagraph{What is regenerative braking? How do the different brake
modes (D, D1, D2, D3, B) effect the driving experience of the
vehicle?}\label{what-is-regenerative-braking-how-do-the-different-brake-modes-d-d1-d2-d3-b-effect-the-driving-experience-of-the-vehicle}}

The e-golf, like all other electric vehicles, uses regenerative braking,
a technology that slows the vehicle down using the electric motor
instead of the breaks. As the vehicle is slowed down, the extra energy
is fed back into the motor and charges the vehicle battery. The brake
modes determine how the car will slow down when coasting.

\hypertarget{how-do-the-different-drive-modes-effect-the-driving-experience-of-the-vehicle}{%
\subparagraph{How do the different drive modes effect the driving
experience of the
vehicle?}\label{how-do-the-different-drive-modes-effect-the-driving-experience-of-the-vehicle}}

\hypertarget{how-fast-is-the-e-golf}{%
\subparagraph{How fast is the e-golf?}\label{how-fast-is-the-e-golf}}

df \%\textgreater{}\%

\hypertarget{how-does-temperature-affect-the-performance-of-the-vehicle-i.e.-range-charge-speed-etc.}{%
\subparagraph{How does temperature affect the performance of the vehicle
(i.e.~range, charge speed,
etc.)?}\label{how-does-temperature-affect-the-performance-of-the-vehicle-i.e.-range-charge-speed-etc.}}

\includegraphics{FAQ_files/figure-latex/unnamed-chunk-5-1.pdf}

According to our data (shown above), there is definately a postitive
relationship between temperature and range. Boom!

\hypertarget{what-is-blue-score}{%
\subparagraph{What is blue score?}\label{what-is-blue-score}}

Blue score is\ldots{}\ldots{}a\ldots{}\ldots{}.

\begin{tabular}{l|r}
\hline
cut(average\_speed, 5) & mean\\
\hline
(3.93,17.2] & 87.81429\\
\hline
(17.2,30.4] & 88.56962\\
\hline
(30.4,43.6] & 85.18421\\
\hline
(43.6,56.8] & 80.80000\\
\hline
(56.8,70.1] & 76.54255\\
\hline
NA & 92.00000\\
\hline
\end{tabular}


\end{document}
