\documentclass[]{article}
\usepackage{lmodern}
\usepackage{amssymb,amsmath}
\usepackage{ifxetex,ifluatex}
\usepackage{fixltx2e} % provides \textsubscript
\ifnum 0\ifxetex 1\fi\ifluatex 1\fi=0 % if pdftex
  \usepackage[T1]{fontenc}
  \usepackage[utf8]{inputenc}
\else % if luatex or xelatex
  \ifxetex
    \usepackage{mathspec}
  \else
    \usepackage{fontspec}
  \fi
  \defaultfontfeatures{Ligatures=TeX,Scale=MatchLowercase}
\fi
% use upquote if available, for straight quotes in verbatim environments
\IfFileExists{upquote.sty}{\usepackage{upquote}}{}
% use microtype if available
\IfFileExists{microtype.sty}{%
\usepackage{microtype}
\UseMicrotypeSet[protrusion]{basicmath} % disable protrusion for tt fonts
}{}
\usepackage[margin=1in]{geometry}
\usepackage{hyperref}
\hypersetup{unicode=true,
            pdfborder={0 0 0},
            breaklinks=true}
\urlstyle{same}  % don't use monospace font for urls
\usepackage{graphicx,grffile}
\makeatletter
\def\maxwidth{\ifdim\Gin@nat@width>\linewidth\linewidth\else\Gin@nat@width\fi}
\def\maxheight{\ifdim\Gin@nat@height>\textheight\textheight\else\Gin@nat@height\fi}
\makeatother
% Scale images if necessary, so that they will not overflow the page
% margins by default, and it is still possible to overwrite the defaults
% using explicit options in \includegraphics[width, height, ...]{}
\setkeys{Gin}{width=\maxwidth,height=\maxheight,keepaspectratio}
\IfFileExists{parskip.sty}{%
\usepackage{parskip}
}{% else
\setlength{\parindent}{0pt}
\setlength{\parskip}{6pt plus 2pt minus 1pt}
}
\setlength{\emergencystretch}{3em}  % prevent overfull lines
\providecommand{\tightlist}{%
  \setlength{\itemsep}{0pt}\setlength{\parskip}{0pt}}
\setcounter{secnumdepth}{0}
% Redefines (sub)paragraphs to behave more like sections
\ifx\paragraph\undefined\else
\let\oldparagraph\paragraph
\renewcommand{\paragraph}[1]{\oldparagraph{#1}\mbox{}}
\fi
\ifx\subparagraph\undefined\else
\let\oldsubparagraph\subparagraph
\renewcommand{\subparagraph}[1]{\oldsubparagraph{#1}\mbox{}}
\fi

%%% Use protect on footnotes to avoid problems with footnotes in titles
\let\rmarkdownfootnote\footnote%
\def\footnote{\protect\rmarkdownfootnote}

%%% Change title format to be more compact
\usepackage{titling}

% Create subtitle command for use in maketitle
\providecommand{\subtitle}[1]{
  \posttitle{
    \begin{center}\large#1\end{center}
    }
}

\setlength{\droptitle}{-2em}

  \title{}
    \pretitle{\vspace{\droptitle}}
  \posttitle{}
    \author{}
    \preauthor{}\postauthor{}
    \date{}
    \predate{}\postdate{}
  

\begin{document}

\hypertarget{ev-data-collection-instructions}{%
\subsection{EV Data collection
instructions}\label{ev-data-collection-instructions}}

\hypertarget{getting-started-and-finishing}{%
\subsubsection{Getting Started and
Finishing}\label{getting-started-and-finishing}}

\begin{itemize}
\item
  Trip Start: The computer should be plugged in to the chords for data
  and charger and then the program launched. Collect the data with
  ignition off except for the range indicator data.
\item
  Trip finish: This data is all collected with the ignition still on.
  The computer should be shut down \& unplugged last (the computer
  screen will go to sleep and may need to be woke to collect temp data)
  The computer will need to be off and charged in a wall outlet
\end{itemize}

\hypertarget{driving-trip-start}{%
\subsubsection{Driving Trip Start}\label{driving-trip-start}}

\begin{itemize}
\item
  Driver \#: (1 - Merl) (2 - Mark) ( 3 - Steve) (4 - Tracy) (5 - Lee) (6
  - Morgan) (7 - Alex) (8 - Allison) (9 - Derek) (10 - Chase) (11 -
  Zach) (12 - Darrell)
\item
  Pre-trip - a walk around and check it was done (video if nec and send
  with file name to
  \href{mailto:mmayer@mhc.ab.ca}{\nolinkurl{mmayer@mhc.ab.ca}})
\item
  Lighting options to use- Dim= dusk or dawn or mainly cloud (very
  little visible sun). Dark= dark. Sunny= mainly sunny (can be cloudy
  but bright).
\item
  Temperatures- incar display for outside collected before key is on
  when drivers door is closed. Inside is collected off surface data
  (please round to closest .5)
\item
  Road conditions- Clear= mainly clear. Blizzard= blizzard. Snowy=
  mainly snow covered. Frosty= frosty. Partly covered = partly snow,
  hard pack or icy
\item
  Range indicator and odometer - displayed on dash (odometer is
  displayed with ignition off)
\item
  Departure - You can set this so the car will be warm and ready when
  you leave (can be used off car battery or when plugged in) Driver 1
  Set for only plugged in (this can only be accesed with ignition on)
\item
  Driving modes - selected on the left side of the shifter plate -
  Settings can be: Normal, Eco and Eco+
\item
  Notes: please feel free to leave any notes, feelings or whatever
\end{itemize}

\hypertarget{driving-trip-finish}{%
\subsubsection{Driving Trip Finish}\label{driving-trip-finish}}

\begin{itemize}
\item
  Post-trip - a walk around and check it was done (video if nec and send
  with file name to
  \href{mailto:mmayer@mhc.ab.ca}{\nolinkurl{mmayer@mhc.ab.ca}})
\item
  Temperatures- incar display for outside collected before key is on
  when drivers door is closed. Inside is collected off surface data
  (please round to closest .5)
\item
  Blue Score also displayed on center infotainment screen
\item
  Brake Mode is done with shifter movement 0= Pulled into drive
  D1,D2,D3= pull shifter to the side left or right to toggle between
  them B=pull shifter back
\item
  Climate control used either Yes or No
\item
  Road conditions:

  \begin{itemize}
  \tightlist
  \item
    City: Most speed limits observed under 70km/hr
  \item
    Highway - Most speed limits observed over 70Km/hr
  \item
    Combine if trip is fairly evenly mixed
  \end{itemize}
\item
  Notes: please indicate if you plug in here and feel free to leave any
  notes, feelings or whatever
\end{itemize}

\hypertarget{charging}{%
\subsubsection{Charging}\label{charging}}

\begin{itemize}
\item
  Location: type of charger and where it was plugged in (the cord with
  the car is a type 1) Provide details in notes on location etc.
\item
  Charge: When the car is plugged in, the estimated time left until
  charge is done will display on the dash, please record that time
\end{itemize}


\end{document}
